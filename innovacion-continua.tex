\documentclass[10pt,a4paper]{report}
\usepackage[latin1]{inputenc}
\usepackage[spanish]{babel}
\author{Juli\'an Mayorga}
\title{Innovaci\'on Continua}
\date{2012}
\begin{document}
\maketitle
\abstract{
\paragraph{Introducci\'on}
El prop\'osito del trabajo es repensar c\'omo actuar a la hora de desarrollar nuevos productos. Se va a describir qu\'e diferencia a una empresa operando bajo extrema incertidumbre con respecto a otra con un modelo de negocios ya definido.
La primera tiene como objetivo aprender sobre c\'omo crear un nuevo modelo de negocios a trav\'es de un nuevo producto, a diferencia de la segunda, cuyo objetivo es maximizar ganancias.
\paragraph{M\'etodo}
El m\'etodo a seguir se centra en maximizar el aprendizaje, esto es debido a que se desea innovar, por lo tanto el objetivo es aprender sobre c\'omo pasar de una visi\'on innovadora a un producto real.
Se va a utilizar un modelo iterativo.
\begin{list}{}{Pasos del modelo a usar}
\item A partir de una idea, \textbf{construir} un producto.
\item A partir del producto, \textbf{medir} m\'etricas clave, espec\'ificas del producto.
\item A partir de las m\'etricas, \textbf{aprender} para reducir la cantidad de incertidumbre, y as\'i acercarse a la resoluci\'on de un problema real.
\end{list}
La metodolog\'ia se centra en transformar c\'omo se construyen y lanzan nuevos productos, reduciendo la cantidad de desperdicio (talento humano, tiempo, etc) en este proceso.
\paragraph{Resultados}
Se van a realizar diferentes ciclos iterativos, cada uno con un desarrollo correspondiente de los cuales se van a tomar nuevos aprendizajes, claves a la hora de innovar.
\paragraph{Discusi\'on}
Se aporta conocimiento te\'orico y pr\'actico a todo el que se enfrenta continuamente a problemas y desea transformarlos en soluciones innovadoras; se ofrece mucho m\'as que un nuevo modelo de trabajo, se trata de responder a la pregunta ''C\'omo podemos aprender m\'as r\'apido sobre qu\'e funciona, y descartar lo que no``.
}

\end{document}